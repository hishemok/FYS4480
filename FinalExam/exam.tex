\documentclass[UKenglish,aspectratio=169,12pt]{beamer}
% \usepackage{unicode-math}
% \setmathfont{latinmodern-math.otf}
% \usepackage[utf8]{inputenc}
% \usepackage{babel, fancyvrb, textcomp, bookmark}
\usepackage{physics}

\usetheme[toc, sectionsep=uioblue2, summary, uiostandard]{UiO}

\newcommand{\Dots}{\ensuremath{\ldots}}
\newcommand{\p}[1]{\texttt{#1}}
% \newcommand{\pb}[1]{\textbf{\p{#1}}}
\newcommand{\pcmd}[1]{\p{\textbackslash #1}}
\newcommand{\penv}[1]{\pcmd{begin}\ppar{#1} \Dots{} \pcmd{end}\ppar{#1}}
\newcommand{\ppar}[1]{\p{\{#1\}}}
\newcommand{\zip}{\textsc{zip}}

\author{Hishem Kløvnes}
% \uioemail{dag@ifi.uio.no}
\title{Final Exam}
\subtitle{Quantum Mechanics for Many-Particle Systems 2024}

\begin{document}


\uiofrontpage[dept={Department of Physics},
  info={Final Exam}]


\section{Introduction}
\subsection{Second Quantization}
\subsection{Wicks Theorem}
\subsection{Particle Hole Formalism}

% \subsection{FCI}
% \subsection{Hartree-Fock}
% \subsection{Many-body perturbation theory}
% \subsection{Couple-cluster theory}

\begin{frame}
    \frametitle{Technical background}
    \begin{itemize}
        \item Second Quantization
        \item Wicks Theorem
        \item Particle Hole Formalism      
    \end{itemize}
\end{frame}

\begin{frame}
    \frametitle{Second Quantization}
    Introducing creation and annihilation operators, simplifies the notation of many-body systems. 
    \[
        \hat{a}^\dagger_i \ket{0} = \ket{i}
    \]
    Here we usually prefer to use indexes $i,j,k\cdots $ to denote states below Fermi-level and $a,b,c\cdots$ to denote states above Fermi-level. 

\end{frame}

\begin{frame}
    \frametitle{Wicks Theorem}
    
    \begin{itemize}
        \item Simplifies long and complicated expressions
        \item Reduces the number of terms in the expansion
        \item Expresses normal ordering in terms of contractions
        \item Usefull tool for diagrammatic representations
    \end{itemize}
\end{frame}

\begin{frame}
    \frametitle{Particle Hole Formalism}
    Given a refrence state $\ket{\Phi_0}$, we often work with operators that create or annihilate particles in this state. This gives rise to the particle-hole formalism.\\
    For instance 1-particle 1-hole $(1p 1h)$, 2-particles 2-holes $(2p 2h)$, etc\dots
    \[
        \ket{\Phi_i^a} = a^\dagger_a a_i \ket{\Phi_0}
    \]
    In essence, we just moved a particle from state $i$ to state $a$.
    This notation is a compact and easy to read way of expressing the state of the system.
\end{frame}

\begin{frame}
    \frametitle{Motivation}
    The main goal today is to understand the different methods used to solve a many-body system.
    \[
        \bra{\Psi_0} \hat{H} \ket{\Psi_0} = E \bra{\Psi_0} \ket{\Psi_0}
    \]
    where:
    \[
        \hat{H} = \hat{H}_0 + \hat{H}_I
    \]
    \[
    \hat{H}_0 = \sum_{\alpha \beta}^{}\bra{ \alpha} h_0 \ket{\beta} \hat{a}^\dagger_\alpha \hat{a}_\beta \quad \hat{H}_I= \frac{1}{4} \sum_{\alpha\beta\delta\gamma}^{}\bra{\alpha\beta} v \ket{\delta\gamma} \hat{a}^\dagger_\alpha \hat{a}^\dagger_\beta \hat{a}_\gamma \hat{a}_\delta
    \]

    This can be a difficult task when applied to a system of many particles, due to the interraction term.
\end{frame}

\section{Full Configuration Interaction}

\begin{frame}
    \frametitle{Full Configuration Interaction}
    \textbf{Introduce approximate wave function}\\
    The simplest choice for many-body wave functions are product wave functions:
    \[
        \Psi(x_1,x_2,\dots ,x_N) \approx \phi_1(x_1)\phi_2(x_2)\dots \phi_N(x_N)
    \]
    Such states are easy to work with, if there aren't any correlations between the single-particle states $\phi_i(x_i)$. 
    However, this does not necessarily imply the antisymmetry of the wave function, which is required by fermions. 
    \[
        \Psi(x_1,x_2,\dots ,x_N) = - \Psi(x_2,x_1,\dots ,x_N)
    \]
\end{frame}

\begin{frame}
    \textbf{Pauli Exclusion Principle}\\
    However, writing on determinantal form:
    \[
        \Psi ( x_1,x_2,x_3,\dots ,x_N) = \frac{1}{\sqrt{N!}} \text{det} 
        \begin{bmatrix} 
         \phi_1 (x_1) & \phi_1(x_2) & \dots & \phi_1(x_N)\\ 
        \phi_2(x_1) & \phi_2 (x_2) & \dots & \phi_2(x_N)\\ 
         \vdots\\ \\
        \phi_N(x_1) & \phi_N(x_2) & \dots & \phi_N(x_N)\\ 
        \end{bmatrix}
    \]
    Pauli exclusion principle, and the antisymmetry of the fermion wave function, is automatically satisfied.
    This forces no two rows/cols to be equal $(\text{det}=0)$. Also it follows that interchanging two rows/cols gives a change of sign.
\end{frame}

\begin{frame}
    \textbf{Change of representations}\\
    Working with large slater determinants, becomes chaotic very quickly (Not good for a many body system...). So we change the representation.
    To each single-particle state $\phi_i(x)$ we associate a creation operator $\hat{a}^{\dagger} _i$ and an annihilation operator $\hat{a} _i$. 

    When acing on the vacuum state $\ket{ 0}$, the creation operator $\hat{a} ^{\dagger} _i$ cause a particle to occupy the single-particle state $\phi_i(x)$:
    \[
    \phi_i(x) \rightarrow \hat{a} ^{\dagger} _i \ket{0}
    \]
    And for multiple creation operators, we can occupy multiple states:
    \[
    \phi_i (x_1) \phi_j(x_2) \phi_k(x_3) \rightarrow \hat{a}^{\dagger}_i \hat{a}^{\dagger}_j \hat{a} ^{\dagger}_k \ket{0} 
    \]
  
\end{frame}

\begin{frame}
    \textbf{Defining our ansatz}\\
    Now we impose antisymmetry, by having the fermion operators satisfy the \textit{anticommutation relations}:
    \[
        \hat{a} ^{\dagger} _i \hat{a} ^{\dagger} _j  + \hat{a}_{j}^{\dagger} \hat{a}_{i}^{\dagger}= [\hat{a}^{\dagger}_i,\hat{a}^{\dagger}_j]_+ = \left\{\hat{a}^{\dagger}_i,\hat{a}^{\dagger}_j\right\} = 0
    \]
    this enforces the Pauli exclusion principle. And thus we are ready to define our ansatz as:
    \[
        \ket{\Phi_0} = \left(\prod_{i\leq F} \hat{a} ^{\dagger} _i\right) \ket{ 0}
    \]
\end{frame}

\begin{frame}
    \textbf{Expanding the exact state}\\
    We use our definitions of particle-hole states to expand the exact state wave function:
    \[
    \ket{\Psi_0} = C_0 \ket{\Phi_0} + \sum_{ai} C_i ^a \ket{\Phi_i ^a } + \sum_{abij}C_{ij}^{ab} \ket{ \Phi_{ij}^{ab}} + \dots = (C_0 + \hat{C} )\ket{\Phi_0}
    \]
    On a more compact form:
    \[
        \ket{\Psi_0} = \sum_{PH} C_H^P \Phi_H^P = \left( \sum_{PH} C_H^P \hat{A}_H^P \right)\ket{\Phi_0} 
    \]
    where H stands for $0,1,\dots,n$ holes and P for $0,1,\dots,n$ particles states. We require normalization, which gives:
    \[
        \bra{  \Psi_0} \ket{\Phi_0} = \sum_{PH} | C_H^P |² = 1 
    \]
\end{frame}

\begin{frame}
    \textbf{Finding the energy}\\
    \[
        E = \bra{ \Psi_0} \hat{H} \ket{\Phi_0} = \sum_{PP' H H'} C^{*P}_H  \bra{ \Phi_H^P } \hat{H} \ket{ \Phi_{H'}^{P'}}C_{H'}^{P'}
    \]
    Normally, this is solved by diagonalization.  A diagonalization is equivalent to finding the variational minimum of:
    \[
        \bra{ \Psi_0} \hat{H} \ket{\Phi_0} - \lambda \bra{ \Psi_0} \ket{ \Phi_0},
    \]
    and after some algebra, we the expression:
    \[
        \sum_{P'H'} \bra{ \Phi_H^P} \hat{H} - E \ket{ \Phi_{H'}^{P'}} = 0
    \]
    all we need to do is solve for the coefficient $C_H^P$.
\end{frame}

\begin{frame}
    \textbf{Observations of the method}\\
    The FCI method is a very powerful method, but it is also very computationally expensive, as the computational cost grows exponentially with the number of particles. However, for smaller systems, this is often used as a benchmark for other methods, as it provides an exact solution to the many-body problem, and does not rely on any other approximations.
\end{frame}

\section{Hartree-Fock}

\begin{frame}
    \frametitle{Hartree-Fock}
    \textbf{Introduction}\\
    Hartree-Fock theory is an algorithm for finding an approximation to the ground state energy with a given Hamiltonian. Using a single particle basis $\psi_\alpha$. Here we try to solve the given eigenvalue problem:
    \[
        \hat{h}^{\text{HF}}\psi_{\alpha} = \epsilon_{\alpha}\psi_{\alpha} \quad , \quad \hat{h}^{\text{HF}} = \hat{t} + \hat{u}_{\text{ext}} + \hat{u}^{\text{HF}}.
    \]

    with $\hat{u}_{\text{ext}}$ being zero for electrons around the nucleus. $\hat{t}$ being kinetic energy and $\hat{u}^{\text{HF}}$ being the Hartree-Fock potential.
\end{frame}

\begin{frame}
    \textbf{Variational principle}\\
    The algorithm tries to find an $\hat{u}^{\text{HF}}$, such that 
    \[
    \langle\hat{H}\rangle = E^{\text{HF}} = \bra{\phi_0}\hat{H}\ket{\phi_0}
    \]
    with $\phi_0$ being some ansatz. 

    The \textbf{variational principle} ensures that 
    \[
    E^{\text{HF}} \ge E_0.
    \]
\end{frame}

\begin{frame}
    \textbf{Purpose}\\
    When calculating the energies of all electrons around a nucleus, we know the large majority of their energies comes from their attraction to the nucleus and their kinetic energy orbiting it. If we wan't a more accurate picture, we must also take into account the interactions between all electrons. This is now a \textit{many-body problem}, with no analytical solution. We therefore have to use an approximation. 
\end{frame}

\begin{frame}
    \frametitle{Method}
    \textbf{Theoretically Convenient}\\
    A simple way to define the Hartree-Fock operator $\hat{h}^{\text{HF}}$ is by splitting it into the sum of the one-body part $\hat{h}_{0}$, direct $V^{d}$ and exchange $V^{e}$ terms. 
    \[
        \hat{h}^{\text{HF}} = \hat{h}_{0}(x_{i}) + \sum_{i=1}^{N} V_{i}^{d}(x_{i}) - \sum_{i=1}^{N} V_{i}^{e}(x_{i})
    \]
    Where $\hat{u}^{HF}$ is the sum of the direct and exchange terms.
\end{frame}

\begin{frame}
    \textbf{Change of basis}\\
    Instead of solving the equation of our previous slide, we can vary the coefficients of our single-particle states $\psi_{\alpha}$, and minimize the energy.  The new single-particle basis will be defined through a unitary transformation. \textit{The unitary transformation keeps the basis orthogonal}:
    \[
        \psi_{p}^{\text{HF}} = \sum_{\lambda} C_{p\lambda}\phi_{\lambda}.
    \]
  where the Greek letters comes from the previous basis, and the Latin from the new basis. 
\end{frame}

\begin{frame}
    \textbf{Minimizing the energy}\\
    As we vary the coefficients, the new Slater determinant $\phi^{\text{HF}}$ varies. This makes $E$ a functional of the slater determinant:
    \[
        E\left[\phi^{\text{HF}}\right] = \sum_{i=1}^{N} \bra{i}\hat{h}\ket{i} + \frac{1}{2} \sum_{i,j = 1}^{N} \bra{ij}\hat{v}\ket{ij}_{AS}. 
    \]
    rewriting with our new basis:
    \[
        E\left[\Psi\right] = \sum_{i=1}^{N}\sum_{\alpha\beta} C^{*}_{i\alpha}C_{i\beta} \bra{\alpha}\hat{h}\ket{\beta} + \frac{1}{2} \sum_{i,j = 1}^{N}\sum_{\alpha\beta\gamma\delta} C^{*}_{i\alpha}C^{*}_{j\beta}C_{i\gamma}C_{j\delta} \bra{\alpha\beta}\hat{v}\ket{\gamma\delta}_{AS}.
    \]
\end{frame}

\begin{frame}
    \textbf{Minimizing the energy}\\
    With our last equation from our previous slide, we can do some variational mathematics to find the minimum of the energy. After that, we are left with a simple eigenvalue problem:
    \[
        \sum_{\beta} h^{\text{HF}}_{\alpha\beta} C_{i\beta} = \epsilon_{i}^{\text{HF}}C_{i\alpha},
    \]
    where:
    \[
        h^{HF}_{\alpha\beta} = \bra{\alpha}\hat{h}\ket{\beta} + \sum_{j}^{N}\sum_{\gamma\delta} C^{*}_{j\gamma}C_{j\delta} \bra{\alpha\gamma}\hat{v}\ket{\beta\delta}_{AS}.
    \]
  
\end{frame}

\begin{frame}
    \textbf{Algorthm}\\
    We start to solve the eigenvalue problem with some guess for the coefficients $C_{i\alpha}$. We then calculate the new coefficients, and iterate until we reach convergence. (Not necessarily convergence, but at least until the difference in eigenvalues is satisfactory small).
  
\end{frame}

\begin{frame}
    \textbf{Observations}\\
    The Hartree-Fock method gives an efficient and approximate solution to the many-body problem. It approximates the many-electron wavefunction as a single Slater determinant, capturing mean-field effects. However, it neglects electron correlation, leading to errors in the total energy. Despite this, HF is computationally efficient and often used as a starting point for more advanced methods, such as Coupled Cluster or Configuration Interaction.
\end{frame}

\section{Many-body perturbation theory}

\begin{frame}
    \frametitle{Many-body perturbation theory}
    \textbf{Introducing the interaction term}\\
    We start by expanding the exact ground state wave function in term of a series of Slater determinants:
    \[
        \ket{\Psi_0} = \ket{\Phi_0} + \sum_{m=1}^\infty C_m \ket{ \Phi_m}
    \]
    where we have assumed that the true ground state is dominated by the solution of the unperturbed problem: 
    \[
        \hat{H}_0 \ket{\Phi_0} = W_0 \ket{\Phi_0}
    \]
\end{frame}
\begin{frame}
\textbf{Introducing the interaction term}\\
  $\ket{\Psi_0}$ is not normalized, however $\bra{ \Phi_0} \ket{\Psi_0} = 1$.
  Writing the total energy, along with the ground state energy:
  \[
    \bra{ \Phi_0}\hat{H} \ket{ \Psi_0} = E \bra{\Phi_0}\ket{\Psi_0}=E
  \]
  \[
    \bra{ \Psi_0}\hat{H}_{0} \ket{ \Phi_0} = W_0 \bra{\Psi_0}\ket{\Phi_0}=W_0
  \]
  we are able to identify the interacion term as:
  \[
    \Delta E = E-W_0 = \bra{ \Phi_0} \hat{H}_I \ket{\Psi_0}
  \]
  This quantity, we refer to as the correlation energy.
\end{frame}

\begin{frame}
    \textbf{Perturbation}\\
     In order to obtain a perturbative expansion, we need to expand the exact wave function in terms of $\hat{H}_I$. We start by expanding the schrödinger equation, and rearringing it:
     \[
        - \hat{H}_0 \ket{ \Psi_0} = (-E + \hat{H}_I) \ket{\Psi_0}
     \]
     adding a term $\omega \ket{\Phi_0}$ and projecting on $\ket{\Phi_0}$ we obtain the following equation:
     \[
        \ket{\Psi_0} = \ket{\Phi_0} + \frac{\hat{Q}}{\omega-\hat{H}_0} (\omega-E+\hat{H}_I)\ket{\Psi_0}
     \]
     the energy dependent variable $\omega$ is not yet defined as it varies for different methods.
\end{frame}

\begin{frame}
    \textbf{Perturbation Equation}\\
    If we take the previous equation and set it up for an iterative scheme, we obtain the following equation:
    \[
        \ket{\Psi_0} = \sum_{i=0}^\infty \left\{\frac{\hat{Q}}{\omega-\hat{H}_0}(\omega-E+\hat{H}_I)\right\}^i \ket{\Phi_0}
    \]
    with $\ket{\Psi_0} = \ket{\Phi_0}$ as an inital quess, and wher $\hat{Q}$ is defined as:
    \[
    \hat{Q} = \sum_{m=1}^{\infty} \ket{\Phi_m}\bra{\Phi_m}
    \]
\end{frame}

\begin{frame}
    \textbf{Correlation Energy}\\
    From the last slide, it is apparent that the iterative formula for the correlation energy is:
    \[
        \Delta E = \sum_{i=0}^\infty \bra{\Phi_0} \hat{H}_I \left\{\frac{\hat{Q}}{\omega-\hat{H}_0}(\omega-E+\hat{H}_I) \right\}^i \ket{\Phi_0}
    \]
\end{frame}

\begin{frame}
    \textbf{Brillouin-Wigner}\\
    The Brillouin-Wigner perturbation method sets $\omega = E$, and we obtain the following expression for the correlation energy:
    \[
        \Delta E = \sum_{i=0}^\infty \bra{\Phi_0} \hat{H}_I \left\{\frac{\hat{Q}}{\omega-\hat{H}_0}(\omega-E+\hat{H}_I) \right\}^i \ket{\Phi_0}=
    \]
    \[
        \bra{ \Phi_0} \left(\hat{H}_I + \hat{H}_I \frac{\hat{Q}}{E-\hat{H}} \hat{H}_I + \hat{H}_I \frac{\hat{Q}}{E-\hat{H}} \hat{H}_I \hat{H}_I \frac{\hat{Q}}{E-\hat{H}} \hat{H}_I +\dots\right)\ket{\Phi_0}
    \]
    This method assumes that we know the exact energy $E$, and is also an implicit method.
\end{frame}

\begin{frame}
    \textbf{Rayleigh-Schrödinger}\\
    The Rayleigh-Schrödinger perturbation method sets $\omega = W_0$, and we obtain the following expression for the correlation energy:
    \[
        \sum_{i=0}^\infty \bra{\Phi_0} \hat{H}_I \left\{\frac{\hat{Q}}{\omega-\hat{H}_0}(\hat{H}_I -\Delta E)\right\}^i \ket{\Phi_0}=
    \]
    which again is an implicit method.
\end{frame}
\begin{frame}
    \textbf{however...}\\
    We can solve $\Delta E$ separately, making it much easier to work with. 
    \[
        \Delta E = \sum_{i=0}^\infty \Delta E^{(i)}
    \]
    ex:
    \[
        \Delta E^{(1)} = \bra{\Phi_0} \hat{H}_I \ket{\Phi_0}
    \]
    \[
        \Delta E^{(2)} = \bra{\Phi_0} \hat{H}_I \frac{\hat{Q}}{W_0-\hat{H}_0} \hat{H}_I \ket{\Phi_0}
    \]
    and so on. And thus we can solve the correlation energy $\Delta E$ with a simple marching scheme.
\end{frame}

\begin{frame}
    \textbf{Observations}\\
    The many-body perturbation theory is a powerful method for solving the many-body problem. We have two different approaches to the perturbation problem.\\
    (BW), which assumes that we know the exact energy $E$, and (RS) assumes a known $W_0$. \\In many cases, both methods can make good approximations to the correlation energy. \\
    The method is built on the assumption that the perturbation is small. \\
    The method is not variational, so adding more terms, does not necessarily give a better approximation.
\end{frame}


\section{Couple-cluster theory}

\begin{frame}
    \frametitle{Couple-cluster theory}

\end{frame}

\end{document}
